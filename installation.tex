\chapter{Installation}
\label{chap:start}

The Legion homepage is \url{https://legion.stanford.edu}.  Here you will find
links to everything associated with the project, including a set of
tutorials that are distinct from this manual.  The Legion software distribution is at
\url{https://github.com/StanfordLegion/legion}.  The distribution has been
tested on Linux and macOS.

To install Legion, open a shell and type:

\begin{lstlisting}[language=bash,style=inline]
git clone https://github.com/StanfordLegion/legion.git
cd legion
mkdir build install
cd build
cmake .. -DCMAKE_INSTALL_PREFIX=$PWD/../install
make install -j4
cd ../..
\end{lstlisting}

This installs Legion into the directory
\lstinline{legion/install}. Note that by default, a debug build is
created. To build a release copy of Legion, add
\lstinline{-DCMAKE_BUILD_Type=Release} the the \lstinline{cmake}
command. (A debug build is strongly recommended when initially
developing with Legion, as it enables a number of checks for correct
usage of Legion APIs, in addition to enabling debug symbols.)

The examples in this manual can then be downloaded and built with:

\begin{lstlisting}[language=bash,style=inline]
git clone https://github.com/StanfordLegion/legion-manual.git
cd legion-manual
mkdir build
cmake ../Examples -DCMAKE_PREFIX_PATH=$PWD/../../legion/install
make -j4
\end{lstlisting}

All of the examples in this manual are included in the build, and will
be located under the \lstinline{legion-manual/build} directory.

\section{Regent}

Regent is the companion programming language for Legion.  Regent provides the same
programming model as the Legion \Cpp\ API, but with a nicer syntax, static checking
of various requirements of Legion programs, and compile-time optimizations.
Instructions for installing Regent are maintained at \url{https://regent-lang.org/install}.



